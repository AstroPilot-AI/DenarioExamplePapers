\documentclass[twocolumn]{aastex631}

\usepackage{amsmath}
\usepackage{multirow}
\usepackage{natbib}
\usepackage{graphicx} 
\usepackage{aas_macros}

\begin{document}

We investigate the relationship between dark matter halo merger tree morphology and halo assembly bias, a key challenge in cosmology, by exploring the feasibility of using Graph Neural Networks (GNNs) to predict an assembly bias proxy from the structural properties of halo merger trees. The original intention was to use Topological Data Analysis (TDA) to guide the GNN architecture and training. Merger trees from cosmological N-body simulations were preprocessed, extracting node features (halo mass, concentration, maximum circular velocity, scale factor) and edge features (merger mass ratio, time difference). However, significant issues arose during the calculation of an assembly bias proxy, resulting in a drastic reduction of the dataset size from 1000 to only 25 merger trees, precluding the TDA component. A Graph Convolutional Network (GCN) was trained on this limited dataset to predict the assembly bias proxy. The GNN model failed to demonstrate any meaningful predictive capability, achieving a high Mean Squared Error and a large negative R-squared score on the test set. These results highlight the critical importance of sufficient data for training complex machine learning models and the challenges associated with reliably extracting assembly bias information from merger trees. While the initial attempt was unsuccessful due to data limitations, the proposed methodology warrants further investigation with larger, more robust datasets and the incorporation of TDA to quantify merger tree morphology.

\end{document}
                