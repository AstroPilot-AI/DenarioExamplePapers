\documentclass[twocolumn]{aastex631}

\usepackage{amsmath}
\usepackage{multirow}
\usepackage{natbib}
\usepackage{graphicx} 
\usepackage{aas_macros}

\begin{document}

\subsection{Summary of findings}

This paper investigated the feasibility of predicting halo assembly bias from the morphology of dark matter halo merger trees using Graph Neural Networks (GNNs). The original approach intended to use Topological Data Analysis (TDA) to guide the GNN architecture and training, but significant data limitations prevented the TDA component from being implemented. Merger trees from cosmological N-body simulations were preprocessed, extracting node and edge features. A Graph Convolutional Network (GCN) was trained on a drastically reduced dataset to predict an assembly bias proxy. The GNN model failed to demonstrate any meaningful predictive capability.

\subsection{Data limitations}

A significant challenge encountered was the drastic reduction in dataset size from 1000 to only 25 merger trees due to issues in calculating a reliable assembly bias proxy. This limitation severely hampered the training and evaluation of the GNN model and precluded the use of Topological Data Analysis (TDA) as initially planned. The inability to compute the assembly bias proxy for the vast majority of trees suggests potential issues with the reliability of identifying the main branch, the definition of "$z=0$" halos, or inconsistencies in the raw data structure itself.

\subsection{GNN performance}

The GNN model, trained on the limited dataset, exhibited poor performance. The model achieved a high Mean Squared Error (MSE) and a large negative R-squared score on the test set, indicating that it failed to capture the underlying relationship between merger tree morphology and the assembly bias proxy. Hyperparameter tuning revealed training instability, with some configurations producing NaN losses. The best-performing model still yielded an MSE significantly higher than the variance of the target variable, suggesting that the model's predictions were far from the true values.

\subsection{Lessons learned and future directions}

The primary conclusion is that the GNN model, constrained by an extremely limited dataset, failed to predict the assembly bias proxy from merger tree morphology. The results highlight the critical importance of sufficient data for training complex machine learning models and the challenges associated with reliably extracting assembly bias information from merger trees. While the initial attempt was unsuccessful due to data limitations, the proposed methodology warrants further investigation with larger, more robust datasets. Future work should focus on addressing the issues encountered in calculating the assembly bias proxy, ensuring a sufficient sample size for training and validation. Furthermore, the incorporation of Topological Data Analysis (TDA) to quantify merger tree morphology remains a promising avenue for future research, potentially providing valuable insights to guide GNN architectures and feature engineering.
\

\end{document}
                