\documentclass[twocolumn]{aastex631}

\usepackage{amsmath}
\usepackage{multirow}
\usepackage{natbib}
\usepackage{graphicx} 
\usepackage{aas_macros}

\begin{document}

This study addressed the challenge of extracting cosmological information from dark matter halo merger trees, specifically aiming to predict the matter density parameter ($\Omega_m$) and the amplitude of matter fluctuations ($\sigma_8$). We hypothesized that graph spectral analysis and diffusion geometry could provide meaningful features from merger trees to predict these cosmological parameters. To investigate this, we analyzed a dataset of 1000 merger trees from N-body simulations.

We extracted features using graph spectral analysis, diffusion map embeddings, and astrophysically-motivated edge properties. These engineered features were then used to train regression models, and their performance was compared against a baseline model using aggregated node features and a Graph Convolutional Network (GCN). A key challenge was the computational cost of eigenvalue decomposition for large graphs, leading to a significant number of missing values that were subsequently imputed.

Our results indicated that the engineered graph spectral and diffusion geometry features performed poorly, likely due to the computational limitations during feature extraction and the subsequent imputation of missing values. Simpler aggregated node features, representing global statistics of halo properties, showed significant predictive power, especially for $\Omega_m$. The GCN model achieved the best overall performance, demonstrating the potential of graph neural networks to automatically learn cosmologically relevant information directly from merger tree structures. The GCN showed a strong predictive capability for $\Omega_m$, while the baseline models also showed a strong correlation with this parameter. For $\sigma_8$, the GCN model again showed the best performance, but with a lower R² than for $\Omega_m$. The baseline models also exhibited a positive but weaker correlation, while the PCA-engineered feature models again showed no discernible correlation.

From these results, we learned that sophisticated graph signal processing techniques, when limited by computational constraints and imputation, may not outperform simpler feature engineering approaches. The success of the aggregated node features suggests that the average halo properties within a merger tree are strongly correlated with $\Omega_m$. Furthermore, the superior performance of the GCN highlights the potential of geometric deep learning to automatically learn cosmologically relevant information from merger tree structures, capturing complex relationships that are difficult to hand-engineer. The study also indicates that merger tree morphology, as characterized by the input node features, is more sensitive to changes in the matter density parameter than to the amplitude of matter fluctuations. This underscores the challenges of manual feature engineering in complex graph data and the promise of geometric deep learning for cosmological inference.

\end{document}
                