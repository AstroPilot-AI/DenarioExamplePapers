\documentclass[twocolumn]{aastex631}

\usepackage{amsmath}
\usepackage{multirow}
\usepackage{natbib}
\usepackage{graphicx} 
\usepackage{aas_macros}

\begin{document}


Understanding the formation and evolution of cosmic structures is a cornerstone of modern cosmology. Dark matter halos, the scaffolding upon which galaxies form, grow through a hierarchical process of mergers and accretion, leaving a trace of their history in the form of merger trees. These trees, representing the ancestral lineage of a halo, are thought to encode valuable information about the underlying cosmological parameters that govern the Universe's evolution, such as the matter density parameter ($\Omega_m$) and the amplitude of matter fluctuations ($\sigma_8$). Therefore, extracting cosmological information from the complex architecture of merger trees is a key challenge in modern cosmology.

However, the inherent complexity and variability of merger trees pose significant analytical hurdles. Merger trees are naturally represented as graphs, with nodes representing halos and edges representing merger events. These graphs vary substantially in size and structure, making it difficult to apply traditional machine learning techniques that require fixed-size feature vectors. Furthermore, the relationships between merger tree morphology and cosmological parameters are likely complex and non-linear, making it difficult to identify the most informative features for cosmological inference. Traditional approaches often rely on hand-engineered features that summarize global properties of the trees, potentially missing subtle but important relationships between the merger history and cosmological parameters.

In this work, we explore the application of graph signal processing techniques, specifically graph spectral analysis and diffusion geometry, to extract cosmologically relevant information from dark matter halo merger trees. Our approach is based on the hypothesis that the spectral properties of the graph Laplacian and the diffusion geometry of merger trees encode robust, interpretable, and computationally efficient features that can be used to predict cosmological parameters. The graph Laplacian provides a way to characterize the connectivity and structure of a graph through its eigenvalues and eigenvectors. Diffusion geometry, on the other hand, provides a way to embed the nodes of a graph into a low-dimensional space that captures the diffusion dynamics on the graph. We hypothesize that the eigenvalues and eigenvectors of the graph Laplacian, as well as the low-dimensional embeddings obtained through diffusion maps, capture the overall connectivity and structural properties of the merger trees in a way that is sensitive to the underlying cosmology.

To test this hypothesis, we analyze a dataset of 1000 merger trees extracted from N-body simulations. We extract features from the graph Laplacian spectrum, diffusion map embeddings, and astrophysically-motivated edge properties, such as the scale factor difference and mass ratio between merging halos. We then train regression models using these features to predict $\Omega_m$ and $\sigma_8$. We compare the performance of these models against two baselines: a model using simpler, aggregated node features and a Graph Convolutional Network (GCN). The first baseline serves to test whether sophisticated graph features outperform simple global statistics of node properties. The GCN serves as a benchmark for assessing the potential of geometric deep learning to automatically learn cosmologically relevant features directly from the merger tree structures, bypassing the need for manual feature engineering. By comparing the performance of these different approaches, we aim to evaluate the effectiveness of graph spectral analysis and diffusion geometry for extracting cosmologically relevant information from dark matter halo merger trees and to compare the performance of these methods to simpler feature engineering approaches and more sophisticated graph neural networks.


\end{document}
                