\documentclass[twocolumn]{aastex631}

\usepackage{amsmath}
\usepackage{multirow}
\usepackage{natbib}
\usepackage{graphicx} 
\usepackage{aas_macros}

\begin{document}

Extracting meaningful features from cosmological merger trees, which encode the hierarchical assembly history of dark matter halos, is crucial for predicting halo properties. This paper explores the use of Quantum Tensor Trains (QTT) for feature engineering on localized subgraphs extracted from merger trees, aiming to predict final halo mass at z=0. QTT is applied to the feature matrix of k-hop neighborhoods around nodes on the main progenitor branch, generating compressed feature vectors representing the local environment. These QTT-informed subgraph features are then used as input to a Random Forest regressor. Using a dataset of 300 merger trees in PyTorch Geometric format, we implemented this approach; however, a significant challenge arose during subgraph extraction, resulting in a severely limited effective sample size of only 5 trees due to invalid node indices. Consequently, while the QTT-derived features showed promising in-sample predictive performance on this limited dataset, these results are not statistically significant or generalizable. This work serves as a proof-of-concept, demonstrating the pipeline's functionality and identifying key challenges, particularly the need for a larger, more representative dataset to rigorously evaluate the potential of QTT-informed feature engineering for merger tree analysis.

\end{document}
                